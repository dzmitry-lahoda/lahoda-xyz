\documentclass[10pt]{article}

\usepackage{url}
\usepackage{xcolor}
\usepackage{tikz}
\usepackage{mdframed}
\usepackage[papersize={6in,9in}, hmargin={.75in, .75in}, top=.25in]{geometry}

\definecolor{bg}{HTML}{265773}
\pagecolor{bg}

\usepackage[T1]{fontenc}
\usepackage{newpxtext}
\usepackage[vvarbb,cmintegrals,cmbraces,bigdelims]{newpxmath}
\usepackage[scr=rsfso]{mathalfa}% \mathscr is fancier than \mathcal
\linespread{1.04}         % adds more leading (space between lines)
% quantifiers look strange, so change those back to normal:
	\DeclareSymbolFont{mysymbols}{OMS}{cmsy}{b}{n} %note we make the figures bold to better match newpx.  Replace the ``b'' with an ``m'' to undo this.
	%\SetSymbolFont{mysymbols}  {bold}{OMS}{cmsy}{b}{n}
	%\DeclareSymbolFont{myoperators}   {OT1}{cmr} {m}{n}
	%\SetSymbolFont{myoperators}{bold}{OT1}{cmr} {bx}{n}
	\DeclareMathSymbol{\forall}{\mathord}{mysymbols}{"38}
	\DeclareMathSymbol{\exists}{\mathord}{mysymbols}{"39}
	%\DeclareMathSymbol{\pm}{\mathbin}{mysymbols}{"06}
	%\DeclareMathSymbol{+}{\mathbin}{myoperators}{"2B}
	%\DeclareMathSymbol{-}{\mathbin}{mysymbols}{"00}
	%\DeclareMathSymbol{=}{\mathrel}{myoperators}{"3D}


\begin{document}

\pagestyle{empty}
~
\begin{center}
\begin{tikzpicture}[yshift=-.75in, scale=.9, remember picture, overlay, color=bg!90]
\def\r{.55}
\newcommand{\hexbox}[3]{
  \def\x{-cos{30}*\r*#1+cos{30}*#2*\r*2}
  \def\y{-\r*#1-sin{30}*\r*#1}
  \draw (\x,\y) +(90:\r) -- +(30:\r) -- +(-30:\r) -- +(-90:\r) -- +(-150:\r) -- +(150:\r) -- cycle;
  \draw (\x,\y) node{#3};
}


% Pascal's triangle
%put row of 1's down left side:
  \foreach \row in {0,...,16} {
    \hexbox{\row}{0}{\large 1}
  }
%fill in the rest of the triangle:
  \foreach \row in {1,...,16} {
    \pgfmathsetmacro{\entry}{1};
    \foreach \col in {1,...,\row} {
      % iterative formula : val = precval * (row-col+1)/col
      % (+ 0.5 to bypass rounding errors)
      \pgfmathtruncatemacro{\entry}{\entry*((\row-\col+1)/\col)+0.5};
      \global\let\entry=\entry
      \ifnum \entry<100
	\hexbox{\row}{\col}{\large \entry}
      \else \ifnum \entry<1000
	\hexbox{\row}{\col}{\entry}
      \else \ifnum \entry<10000
	\hexbox{\row}{\col}{\footnotesize \entry}
	\else
	\hexbox{\row}{\col}{\scriptsize \entry}
	\fi
      \fi
      \fi
    }
  }
  % \foreach \row in{17,...,20} {
  % \foreach \col in {1,...,\row} {
  % \hexbox{\row}{\col}{}
  % }
  % }
\end{tikzpicture}
\end{center}
\hspace{-3em}{\color{bg!65} Mathematics}


\vskip 1.5in
\noindent This gentle introduction to discrete mathematics is written for first and second year math majors, especially those who intend to teach.  The text began as a set of lecture notes for the discrete mathematics course at the University of Northern Colorado.  This course serves both as an introduction to topics in discrete math and as the ``introduction to proof'' course for math majors.  The course is usually taught with a large amount of student inquiry, and this text is written to help facilitate this.

\vskip 2em

\noindent Four main topics are covered: counting, sequences, logic, and graph theory.  Along the way proofs are introduced, including proofs by contradiction, proofs by induction, and combinatorial proofs.

\vskip 2em

\noindent While there are many fine discrete math textbooks available, this text has the following advantages:
\begin{itemize}
\item[--] It is written to be used in an inquiry rich course.
\item[--] It is written to be used in a course for future math teachers.
\item[--] It is open source, with low cost print editions and free electronic editions.
\end{itemize}

\vskip 5em
\begin{center}To download the current version, or for information on obtaining the MathBook XML source, visit: \\\url{http://discretetext.oscarlevin.com/}.
\end{center}


\clearpage

\end{document}
